% Options for packages loaded elsewhere
\PassOptionsToPackage{unicode}{hyperref}
\PassOptionsToPackage{hyphens}{url}
%
\documentclass[
]{article}
\usepackage{amsmath,amssymb}
\usepackage{iftex}
\ifPDFTeX
  \usepackage[T1]{fontenc}
  \usepackage[utf8]{inputenc}
  \usepackage{textcomp} % provide euro and other symbols
\else % if luatex or xetex
  \usepackage{unicode-math} % this also loads fontspec
  \defaultfontfeatures{Scale=MatchLowercase}
  \defaultfontfeatures[\rmfamily]{Ligatures=TeX,Scale=1}
\fi
\usepackage{lmodern}
\ifPDFTeX\else
  % xetex/luatex font selection
\fi
% Use upquote if available, for straight quotes in verbatim environments
\IfFileExists{upquote.sty}{\usepackage{upquote}}{}
\IfFileExists{microtype.sty}{% use microtype if available
  \usepackage[]{microtype}
  \UseMicrotypeSet[protrusion]{basicmath} % disable protrusion for tt fonts
}{}
\makeatletter
\@ifundefined{KOMAClassName}{% if non-KOMA class
  \IfFileExists{parskip.sty}{%
    \usepackage{parskip}
  }{% else
    \setlength{\parindent}{0pt}
    \setlength{\parskip}{6pt plus 2pt minus 1pt}}
}{% if KOMA class
  \KOMAoptions{parskip=half}}
\makeatother
\usepackage{xcolor}
\usepackage[margin=1in]{geometry}
\usepackage{color}
\usepackage{fancyvrb}
\newcommand{\VerbBar}{|}
\newcommand{\VERB}{\Verb[commandchars=\\\{\}]}
\DefineVerbatimEnvironment{Highlighting}{Verbatim}{commandchars=\\\{\}}
% Add ',fontsize=\small' for more characters per line
\usepackage{framed}
\definecolor{shadecolor}{RGB}{248,248,248}
\newenvironment{Shaded}{\begin{snugshade}}{\end{snugshade}}
\newcommand{\AlertTok}[1]{\textcolor[rgb]{0.94,0.16,0.16}{#1}}
\newcommand{\AnnotationTok}[1]{\textcolor[rgb]{0.56,0.35,0.01}{\textbf{\textit{#1}}}}
\newcommand{\AttributeTok}[1]{\textcolor[rgb]{0.13,0.29,0.53}{#1}}
\newcommand{\BaseNTok}[1]{\textcolor[rgb]{0.00,0.00,0.81}{#1}}
\newcommand{\BuiltInTok}[1]{#1}
\newcommand{\CharTok}[1]{\textcolor[rgb]{0.31,0.60,0.02}{#1}}
\newcommand{\CommentTok}[1]{\textcolor[rgb]{0.56,0.35,0.01}{\textit{#1}}}
\newcommand{\CommentVarTok}[1]{\textcolor[rgb]{0.56,0.35,0.01}{\textbf{\textit{#1}}}}
\newcommand{\ConstantTok}[1]{\textcolor[rgb]{0.56,0.35,0.01}{#1}}
\newcommand{\ControlFlowTok}[1]{\textcolor[rgb]{0.13,0.29,0.53}{\textbf{#1}}}
\newcommand{\DataTypeTok}[1]{\textcolor[rgb]{0.13,0.29,0.53}{#1}}
\newcommand{\DecValTok}[1]{\textcolor[rgb]{0.00,0.00,0.81}{#1}}
\newcommand{\DocumentationTok}[1]{\textcolor[rgb]{0.56,0.35,0.01}{\textbf{\textit{#1}}}}
\newcommand{\ErrorTok}[1]{\textcolor[rgb]{0.64,0.00,0.00}{\textbf{#1}}}
\newcommand{\ExtensionTok}[1]{#1}
\newcommand{\FloatTok}[1]{\textcolor[rgb]{0.00,0.00,0.81}{#1}}
\newcommand{\FunctionTok}[1]{\textcolor[rgb]{0.13,0.29,0.53}{\textbf{#1}}}
\newcommand{\ImportTok}[1]{#1}
\newcommand{\InformationTok}[1]{\textcolor[rgb]{0.56,0.35,0.01}{\textbf{\textit{#1}}}}
\newcommand{\KeywordTok}[1]{\textcolor[rgb]{0.13,0.29,0.53}{\textbf{#1}}}
\newcommand{\NormalTok}[1]{#1}
\newcommand{\OperatorTok}[1]{\textcolor[rgb]{0.81,0.36,0.00}{\textbf{#1}}}
\newcommand{\OtherTok}[1]{\textcolor[rgb]{0.56,0.35,0.01}{#1}}
\newcommand{\PreprocessorTok}[1]{\textcolor[rgb]{0.56,0.35,0.01}{\textit{#1}}}
\newcommand{\RegionMarkerTok}[1]{#1}
\newcommand{\SpecialCharTok}[1]{\textcolor[rgb]{0.81,0.36,0.00}{\textbf{#1}}}
\newcommand{\SpecialStringTok}[1]{\textcolor[rgb]{0.31,0.60,0.02}{#1}}
\newcommand{\StringTok}[1]{\textcolor[rgb]{0.31,0.60,0.02}{#1}}
\newcommand{\VariableTok}[1]{\textcolor[rgb]{0.00,0.00,0.00}{#1}}
\newcommand{\VerbatimStringTok}[1]{\textcolor[rgb]{0.31,0.60,0.02}{#1}}
\newcommand{\WarningTok}[1]{\textcolor[rgb]{0.56,0.35,0.01}{\textbf{\textit{#1}}}}
\usepackage{graphicx}
\makeatletter
\def\maxwidth{\ifdim\Gin@nat@width>\linewidth\linewidth\else\Gin@nat@width\fi}
\def\maxheight{\ifdim\Gin@nat@height>\textheight\textheight\else\Gin@nat@height\fi}
\makeatother
% Scale images if necessary, so that they will not overflow the page
% margins by default, and it is still possible to overwrite the defaults
% using explicit options in \includegraphics[width, height, ...]{}
\setkeys{Gin}{width=\maxwidth,height=\maxheight,keepaspectratio}
% Set default figure placement to htbp
\makeatletter
\def\fps@figure{htbp}
\makeatother
\setlength{\emergencystretch}{3em} % prevent overfull lines
\providecommand{\tightlist}{%
  \setlength{\itemsep}{0pt}\setlength{\parskip}{0pt}}
\setcounter{secnumdepth}{-\maxdimen} % remove section numbering
\ifLuaTeX
  \usepackage{selnolig}  % disable illegal ligatures
\fi
\usepackage{bookmark}
\IfFileExists{xurl.sty}{\usepackage{xurl}}{} % add URL line breaks if available
\urlstyle{same}
\hypersetup{
  pdftitle={Statistical Learning in Movies},
  pdfauthor={Max Sellers, Claire Martino, Ari Augustine, and Rim Nassiri},
  hidelinks,
  pdfcreator={LaTeX via pandoc}}

\title{Statistical Learning in Movies}
\author{Max Sellers, Claire Martino, Ari Augustine, and Rim Nassiri}
\date{}

\begin{document}
\maketitle

\subsubsection{Abstract}\label{abstract}

To be a producer of film, understanding the factors influencing movie
success is extremely important. This project uses statistical learning
techniques to predict how successful a film will be critically and
popularity-wise. Using a dataset compiled from the Internet Movie
Database (IMDB), we delve into the relationships between these variables
and movie ratings. Our goal in this project is to use variables like the
movie's budget, revenue, runtime, rating, and popularity we can
determine what the film's critical and commercial success could be by
using models like simple and multiple linear regression.

\subsection{I. Introduction}\label{i.-introduction}

In the dynamic film industry, understanding the factors that contribute
to a movie's success can be unpredictable. Our data project examines a
movie dataset, aiming to uncover relationships between key variables and
a movie's performance. This analysis is motivated by an interest in
understanding if there are underlying patterns that govern movie
success. By using statistical models, particularly linear regression, we
aim to find out how significant each variable is in affecting a movie's
success.\\
The movie dataset utilized in this project has a wide range of films
spanning different genres, release years, and production scales. Each
movie entry in the dataset includes information on key variables
providing insights into various aspects such as financial investment and
audience engagement. The variables we've selected for further analysis
include revenue, budget, popularity, rating, and runtime. By analyzing
the data, we aim to identify patterns and trends driving success in the
film industry. In our initial exploration of the dataset, we have
formulated hypotheses regarding the potential impact of these
variables---budget, popularity, rating, and runtime---on a movie's
revenue. These hypotheses are based on common assumptions and consumer
behavior. We anticipate that a larger budget will be positively
correlated with higher revenue. Greater investment typically enables
higher production values, marketing campaigns, and wider distribution,
all of which contribute to higher returns. Additionally, higher levels
of popularity and rating are predicted to correlate positively with
revenue since an increased level of interest usually results in more
sales. There may be a non-linear relationship between movie runtime and
revenue since it's difficult to anticipate and may not have substantial
significance. Through exploratory data analysis and model building, we
hope to gain deeper insights into the factors influencing a movie's
revenue.

\subsection{II. Methods}\label{ii.-methods}

\subsubsection{a. Cleaning Data}\label{a.-cleaning-data}

\begin{Shaded}
\begin{Highlighting}[]
\NormalTok{movies\_data }\OtherTok{=}\NormalTok{ movies\_data }\SpecialCharTok{\%\textgreater{}\%}
  \FunctionTok{group\_by}\NormalTok{(title) }\SpecialCharTok{\%\textgreater{}\%}
  \FunctionTok{filter}\NormalTok{(budget }\SpecialCharTok{\textgreater{}=} \DecValTok{1000000} \SpecialCharTok{\&}\NormalTok{ revenue }\SpecialCharTok{\textgreater{}=} \DecValTok{1000000} \SpecialCharTok{\&}\NormalTok{ vote\_count }\SpecialCharTok{\textgreater{}} \DecValTok{100}\NormalTok{) }\SpecialCharTok{\%\textgreater{}\%}
  \FunctionTok{arrange}\NormalTok{(}\FunctionTok{desc}\NormalTok{(vote\_average))}
\end{Highlighting}
\end{Shaded}

\subsubsection{b. Choosing Important Values on
Dataset}\label{b.-choosing-important-values-on-dataset}

\begin{Shaded}
\begin{Highlighting}[]
\CommentTok{\#number of categories and variables in the dataset}
\NormalTok{categories }\OtherTok{=} \FunctionTok{ncol}\NormalTok{(movies\_data)}
\NormalTok{var }\OtherTok{=} \FunctionTok{nrow}\NormalTok{(movies\_data)}

\CommentTok{\#means of variables that will be using in analysis}
\NormalTok{mean\_budget }\OtherTok{=} \FunctionTok{format}\NormalTok{(}\FunctionTok{round}\NormalTok{(}\FunctionTok{mean}\NormalTok{(movies\_data}\SpecialCharTok{$}\NormalTok{budget),}\DecValTok{2}\NormalTok{),}\AttributeTok{scientific=}\NormalTok{F)}
\NormalTok{mean\_popularity }\OtherTok{=} \FunctionTok{format}\NormalTok{(}\FunctionTok{round}\NormalTok{(}\FunctionTok{mean}\NormalTok{(movies\_data}\SpecialCharTok{$}\NormalTok{popularity),}\DecValTok{2}\NormalTok{), }\AttributeTok{scientific =}\NormalTok{ F)}
\NormalTok{mean\_revenue }\OtherTok{=} \FunctionTok{format}\NormalTok{(}\FunctionTok{round}\NormalTok{(}\FunctionTok{mean}\NormalTok{(movies\_data}\SpecialCharTok{$}\NormalTok{revenue),}\DecValTok{2}\NormalTok{), }\AttributeTok{scientific =}\NormalTok{ F)}
\NormalTok{mean\_vote\_average }\OtherTok{=} \FunctionTok{format}\NormalTok{(}\FunctionTok{round}\NormalTok{(}\FunctionTok{mean}\NormalTok{(movies\_data}\SpecialCharTok{$}\NormalTok{vote\_average),}\DecValTok{2}\NormalTok{), }\AttributeTok{scientific =}\NormalTok{ F)}
\NormalTok{mean\_runtime }\OtherTok{=} \FunctionTok{format}\NormalTok{(}\FunctionTok{round}\NormalTok{(}\FunctionTok{mean}\NormalTok{(movies\_data}\SpecialCharTok{$}\NormalTok{runtime),}\DecValTok{2}\NormalTok{), }\AttributeTok{scientific =}\NormalTok{ F)}
\end{Highlighting}
\end{Shaded}

There are 12 categories in this movies dataset that we will be using.
There are 3566 rows in this dataset that we will be using.

The mean budget of the dataset is 42149142. The mean popularity of the
movies is 12.34. The mean revenue of the movies is 130185255. The mean
voter average of the movies 6.39. The mean runtime of the movies is
110.84.

\subsubsection{c.~Training/Testing Data {[}USE
SOMEWHERE{]}}\label{c.-trainingtesting-data-use-somewhere}

\begin{Shaded}
\begin{Highlighting}[]
\NormalTok{num\_obs }\OtherTok{=} \FunctionTok{nrow}\NormalTok{(movies\_data)  }\CommentTok{\#extracting the total rows from the data set}

\NormalTok{movies\_indx }\OtherTok{=} \FunctionTok{sample}\NormalTok{(num\_obs, }\AttributeTok{size =} \FunctionTok{trunc}\NormalTok{(}\FloatTok{0.5}\SpecialCharTok{*}\NormalTok{num\_obs))}

\CommentTok{\#training data set:}
\NormalTok{movies\_train }\OtherTok{=}\NormalTok{ movies\_data[movies\_indx,]}

\CommentTok{\#test data set:}
\NormalTok{movies\_test }\OtherTok{=}\NormalTok{ movies\_data[}\SpecialCharTok{{-}}\NormalTok{movies\_indx,]}
\end{Highlighting}
\end{Shaded}

\subsection{III. Results}\label{iii.-results}

\subsubsection{a. Linear Regression Summary and
Line}\label{a.-linear-regression-summary-and-line}

\begin{Shaded}
\begin{Highlighting}[]
\CommentTok{\#linear regression model of budget vs vote\_average}
\NormalTok{linear\_1 }\OtherTok{=} \FunctionTok{lm}\NormalTok{(budget }\SpecialCharTok{\textasciitilde{}}\NormalTok{ vote\_average, }\AttributeTok{data =}\NormalTok{ movies\_data)}
\NormalTok{summary\_linear1 }\OtherTok{=} \FunctionTok{summary}\NormalTok{(linear\_1)}

\NormalTok{r\_2\_lin\_1 }\OtherTok{=}\NormalTok{ summary\_linear1}\SpecialCharTok{$}\NormalTok{r.squared}


\CommentTok{\#graph of above data}
\FunctionTok{plot}\NormalTok{(budget}\SpecialCharTok{\textasciitilde{}}\NormalTok{ vote\_average, }\AttributeTok{data=}\NormalTok{ movies\_data, }\AttributeTok{xlab =} \StringTok{"Voter Average"}\NormalTok{, }\AttributeTok{ylab =} \StringTok{"Budget"}\NormalTok{, }\AttributeTok{main =} \StringTok{"Scatterplot of Budget vs Voter Average"}\NormalTok{)}
\FunctionTok{abline}\NormalTok{(linear\_1, }\AttributeTok{col =} \StringTok{"red"}\NormalTok{)}
\end{Highlighting}
\end{Shaded}

\includegraphics{IMDB_files/figure-latex/unnamed-chunk-6-1.pdf}

Since the Pr(\textgreater\textbar t\textbar) value of voter average is
8.77e\_07, and this value is less than the standard level of
significance of 0.05, this shows that there is a statistically
significant relationship between voter average and budget.

In order to assess the relationship between the predictor and the
response variable, you must look at the \(R^2\) value. In this case,
\(R^2\) = 0.006763. Since this value is closer to 0 than it is 1, this
indicates a weak relationship between voter average and budget.

This scatterplot with the linear regression line shows a weak and
negative relationship between between budget and voter average.

\begin{Shaded}
\begin{Highlighting}[]
\CommentTok{\#linear regression model of revenue vs popularity}
\NormalTok{linear\_2 }\OtherTok{=} \FunctionTok{lm}\NormalTok{(revenue }\SpecialCharTok{\textasciitilde{}}\NormalTok{ popularity, }\AttributeTok{data =}\NormalTok{ movies\_data)}
\NormalTok{summary\_linear2 }\OtherTok{=} \FunctionTok{summary}\NormalTok{(linear\_2)}

\NormalTok{r\_2\_lin\_2 }\OtherTok{=}\NormalTok{ summary\_linear2}\SpecialCharTok{$}\NormalTok{r.squared}


\FunctionTok{plot}\NormalTok{(revenue }\SpecialCharTok{\textasciitilde{}}\NormalTok{ popularity, }\AttributeTok{data=}\NormalTok{ movies\_data, }\AttributeTok{xlab =} \StringTok{"Popularity"}\NormalTok{, }\AttributeTok{ylab =} \StringTok{"Revenue"}\NormalTok{, }\AttributeTok{main =} \StringTok{"Scatterplot of Revenue vs Popularity"}\NormalTok{)}
\FunctionTok{abline}\NormalTok{(linear\_2, }\AttributeTok{col =} \StringTok{"purple"}\NormalTok{)}
\end{Highlighting}
\end{Shaded}

\includegraphics{IMDB_files/figure-latex/unnamed-chunk-7-1.pdf}

Since the Pr(\textgreater\textbar t\textbar) value of popularity is
\textless2e-16, and this value is less than the standard level of
significance of 0.05, this shows that there is a statistically
significant relationship between popularity and revenue.

In order to assess the relationship between the predictor and the
response variable, you must look at the \(R^2\) value. In this case,
\(R^2\) = 0.1547861. Since this value is closer to 0 than it is 1, this
indicates a mildly weak relationship between revenue and popularity.

The scatter plot with the linear regression line shows this mildy weak
positive relationship between revenue and popularity.

\begin{Shaded}
\begin{Highlighting}[]
\NormalTok{linear\_3 }\OtherTok{=} \FunctionTok{lm}\NormalTok{(vote\_average }\SpecialCharTok{\textasciitilde{}}\NormalTok{ popularity, }\AttributeTok{data =}\NormalTok{ movies\_data)}
\NormalTok{summary\_linear3 }\OtherTok{=} \FunctionTok{summary}\NormalTok{(linear\_3)}

\NormalTok{r\_2\_lin\_3 }\OtherTok{=}\NormalTok{ summary\_linear3}\SpecialCharTok{$}\NormalTok{r.squared}

\FunctionTok{plot}\NormalTok{(vote\_average }\SpecialCharTok{\textasciitilde{}}\NormalTok{ popularity, }\AttributeTok{data=}\NormalTok{ movies\_data, }\AttributeTok{xlab =} \StringTok{"Popularity"}\NormalTok{, }\AttributeTok{ylab =} \StringTok{"Rating"}\NormalTok{, }\AttributeTok{main =} \StringTok{"Scatterplot of Rating vs Popularity"}\NormalTok{)}
\FunctionTok{abline}\NormalTok{(linear\_3, }\AttributeTok{col =} \StringTok{"purple"}\NormalTok{)}
\end{Highlighting}
\end{Shaded}

\includegraphics{IMDB_files/figure-latex/unnamed-chunk-8-1.pdf}

Since the Pr(\textgreater\textbar t\textbar) value of popularity is
8.77e-07, and this value is less than the standard level of significance
of 0.05, this shows that there is a statistically significant
relationship between popularity and rating.

In order to assess the relationship between the predictor and the
response variable, you must look at the \(R^2\) value. In this case,
\(R^2\) = 0.0185173. Since this value is closer to 0 than it is 1, this
indicates a weak relationship between rating and popularity.

The scatter plot with the linear regression line shows this weak
positive relationship between rating and popularity.

\subsubsection{b. Trained Simple Linear Regression
Summary}\label{b.-trained-simple-linear-regression-summary}

\begin{Shaded}
\begin{Highlighting}[]
\NormalTok{y.trn }\OtherTok{=}\NormalTok{ movies\_train}\SpecialCharTok{$}\NormalTok{revenue}
\NormalTok{x.trn }\OtherTok{=}\NormalTok{ movies\_train}\SpecialCharTok{$}\NormalTok{budget}

\NormalTok{train.data\_frame }\OtherTok{=} \FunctionTok{data.frame}\NormalTok{(y.trn, x.trn)}

\NormalTok{model\_train }\OtherTok{=} \FunctionTok{glm}\NormalTok{(}\AttributeTok{formula =}\NormalTok{ y.trn }\SpecialCharTok{\textasciitilde{}}\NormalTok{ x.trn, }\AttributeTok{data =}\NormalTok{ train.data\_frame)}
\FunctionTok{summary}\NormalTok{(model\_train)}
\end{Highlighting}
\end{Shaded}

\begin{verbatim}
## 
## Call:
## glm(formula = y.trn ~ x.trn, data = train.data_frame)
## 
## Coefficients:
##              Estimate Std. Error t value Pr(>|t|)    
## (Intercept) 4.569e+06  4.492e+06   1.017    0.309    
## x.trn       2.976e+00  7.355e-02  40.456   <2e-16 ***
## ---
## Signif. codes:  0 '***' 0.001 '**' 0.01 '*' 0.05 '.' 0.1 ' ' 1
## 
## (Dispersion parameter for gaussian family taken to be 1.937784e+16)
## 
##     Null deviance: 6.6227e+19  on 1782  degrees of freedom
## Residual deviance: 3.4512e+19  on 1781  degrees of freedom
## AIC: 71932
## 
## Number of Fisher Scoring iterations: 2
\end{verbatim}

\begin{Shaded}
\begin{Highlighting}[]
\FunctionTok{plot}\NormalTok{(model\_train)}
\end{Highlighting}
\end{Shaded}

\includegraphics{IMDB_files/figure-latex/unnamed-chunk-9-1.pdf}
\includegraphics{IMDB_files/figure-latex/unnamed-chunk-9-2.pdf}
\includegraphics{IMDB_files/figure-latex/unnamed-chunk-9-3.pdf}
\includegraphics{IMDB_files/figure-latex/unnamed-chunk-9-4.pdf}

\begin{Shaded}
\begin{Highlighting}[]
\NormalTok{actual }\OtherTok{=} \FunctionTok{ifelse}\NormalTok{(movies\_test}\SpecialCharTok{$}\NormalTok{revenue}\SpecialCharTok{\textgreater{}} \DecValTok{100000000}\NormalTok{, }\StringTok{"Yes"}\NormalTok{, }\StringTok{"No"}\NormalTok{)}
\FunctionTok{head}\NormalTok{(actual)}
\end{Highlighting}
\end{Shaded}

\begin{verbatim}
## [1] "No"  "No"  "Yes" "Yes" "No"  "Yes"
\end{verbatim}

\begin{Shaded}
\begin{Highlighting}[]
\NormalTok{predicted }\OtherTok{=} \FunctionTok{ifelse}\NormalTok{(}\FunctionTok{predict}\NormalTok{(model\_train) }\SpecialCharTok{\textgreater{}} \DecValTok{100000000}\NormalTok{, }\StringTok{"Yes"}\NormalTok{, }\StringTok{"No"}\NormalTok{)}
\FunctionTok{head}\NormalTok{(predicted)}
\end{Highlighting}
\end{Shaded}

\begin{verbatim}
##     1     2     3     4     5     6 
##  "No"  "No" "Yes" "Yes"  "No" "Yes"
\end{verbatim}

\begin{Shaded}
\begin{Highlighting}[]
\NormalTok{error\_function }\OtherTok{=} \ControlFlowTok{function}\NormalTok{(actual, predicted)\{}
  \FunctionTok{mean}\NormalTok{(actual }\SpecialCharTok{!=}\NormalTok{ predicted)}
\NormalTok{\}}

\FunctionTok{error\_function}\NormalTok{(}\AttributeTok{actual =}\NormalTok{ actual, }\AttributeTok{predicted =}\NormalTok{ predicted)}
\end{Highlighting}
\end{Shaded}

\begin{verbatim}
## [1] 0.4756029
\end{verbatim}

\begin{Shaded}
\begin{Highlighting}[]
\DecValTok{1} \SpecialCharTok{{-}} \FunctionTok{error\_function}\NormalTok{(}\AttributeTok{actual =}\NormalTok{ actual, }\AttributeTok{predicted =}\NormalTok{ predicted)}
\end{Highlighting}
\end{Shaded}

\begin{verbatim}
## [1] 0.5243971
\end{verbatim}

\begin{Shaded}
\begin{Highlighting}[]
\FunctionTok{library}\NormalTok{(caret)}
\end{Highlighting}
\end{Shaded}

\begin{verbatim}
## Warning: package 'caret' was built under R version 4.3.2
\end{verbatim}

\begin{verbatim}
## Loading required package: lattice
\end{verbatim}

\begin{Shaded}
\begin{Highlighting}[]
\NormalTok{train\_table }\OtherTok{=} \FunctionTok{table}\NormalTok{(}\AttributeTok{predicted =}\NormalTok{ predicted, }\AttributeTok{Actual =}\NormalTok{ actual)}

\NormalTok{train\_confusion\_matrix }\OtherTok{=} \FunctionTok{confusionMatrix}\NormalTok{(train\_table, }\AttributeTok{positive =} \StringTok{"Yes"}\NormalTok{)}

\NormalTok{train\_confusion\_matrix}
\end{Highlighting}
\end{Shaded}

\begin{verbatim}
## Confusion Matrix and Statistics
## 
##          Actual
## predicted  No Yes
##       No  650 391
##       Yes 457 285
##                                           
##                Accuracy : 0.5244          
##                  95% CI : (0.5009, 0.5478)
##     No Information Rate : 0.6209          
##     P-Value [Acc > NIR] : 1.00000         
##                                           
##                   Kappa : 0.0086          
##                                           
##  Mcnemar's Test P-Value : 0.02561         
##                                           
##             Sensitivity : 0.4216          
##             Specificity : 0.5872          
##          Pos Pred Value : 0.3841          
##          Neg Pred Value : 0.6244          
##              Prevalence : 0.3791          
##          Detection Rate : 0.1598          
##    Detection Prevalence : 0.4162          
##       Balanced Accuracy : 0.5044          
##                                           
##        'Positive' Class : Yes             
## 
\end{verbatim}

\subsection{IV. Discussion}\label{iv.-discussion}

\end{document}
