% Options for packages loaded elsewhere
\PassOptionsToPackage{unicode}{hyperref}
\PassOptionsToPackage{hyphens}{url}
%
\documentclass[
]{article}
\usepackage{amsmath,amssymb}
\usepackage{iftex}
\ifPDFTeX
  \usepackage[T1]{fontenc}
  \usepackage[utf8]{inputenc}
  \usepackage{textcomp} % provide euro and other symbols
\else % if luatex or xetex
  \usepackage{unicode-math} % this also loads fontspec
  \defaultfontfeatures{Scale=MatchLowercase}
  \defaultfontfeatures[\rmfamily]{Ligatures=TeX,Scale=1}
\fi
\usepackage{lmodern}
\ifPDFTeX\else
  % xetex/luatex font selection
\fi
% Use upquote if available, for straight quotes in verbatim environments
\IfFileExists{upquote.sty}{\usepackage{upquote}}{}
\IfFileExists{microtype.sty}{% use microtype if available
  \usepackage[]{microtype}
  \UseMicrotypeSet[protrusion]{basicmath} % disable protrusion for tt fonts
}{}
\makeatletter
\@ifundefined{KOMAClassName}{% if non-KOMA class
  \IfFileExists{parskip.sty}{%
    \usepackage{parskip}
  }{% else
    \setlength{\parindent}{0pt}
    \setlength{\parskip}{6pt plus 2pt minus 1pt}}
}{% if KOMA class
  \KOMAoptions{parskip=half}}
\makeatother
\usepackage{xcolor}
\usepackage[margin=1in]{geometry}
\usepackage{color}
\usepackage{fancyvrb}
\newcommand{\VerbBar}{|}
\newcommand{\VERB}{\Verb[commandchars=\\\{\}]}
\DefineVerbatimEnvironment{Highlighting}{Verbatim}{commandchars=\\\{\}}
% Add ',fontsize=\small' for more characters per line
\usepackage{framed}
\definecolor{shadecolor}{RGB}{248,248,248}
\newenvironment{Shaded}{\begin{snugshade}}{\end{snugshade}}
\newcommand{\AlertTok}[1]{\textcolor[rgb]{0.94,0.16,0.16}{#1}}
\newcommand{\AnnotationTok}[1]{\textcolor[rgb]{0.56,0.35,0.01}{\textbf{\textit{#1}}}}
\newcommand{\AttributeTok}[1]{\textcolor[rgb]{0.13,0.29,0.53}{#1}}
\newcommand{\BaseNTok}[1]{\textcolor[rgb]{0.00,0.00,0.81}{#1}}
\newcommand{\BuiltInTok}[1]{#1}
\newcommand{\CharTok}[1]{\textcolor[rgb]{0.31,0.60,0.02}{#1}}
\newcommand{\CommentTok}[1]{\textcolor[rgb]{0.56,0.35,0.01}{\textit{#1}}}
\newcommand{\CommentVarTok}[1]{\textcolor[rgb]{0.56,0.35,0.01}{\textbf{\textit{#1}}}}
\newcommand{\ConstantTok}[1]{\textcolor[rgb]{0.56,0.35,0.01}{#1}}
\newcommand{\ControlFlowTok}[1]{\textcolor[rgb]{0.13,0.29,0.53}{\textbf{#1}}}
\newcommand{\DataTypeTok}[1]{\textcolor[rgb]{0.13,0.29,0.53}{#1}}
\newcommand{\DecValTok}[1]{\textcolor[rgb]{0.00,0.00,0.81}{#1}}
\newcommand{\DocumentationTok}[1]{\textcolor[rgb]{0.56,0.35,0.01}{\textbf{\textit{#1}}}}
\newcommand{\ErrorTok}[1]{\textcolor[rgb]{0.64,0.00,0.00}{\textbf{#1}}}
\newcommand{\ExtensionTok}[1]{#1}
\newcommand{\FloatTok}[1]{\textcolor[rgb]{0.00,0.00,0.81}{#1}}
\newcommand{\FunctionTok}[1]{\textcolor[rgb]{0.13,0.29,0.53}{\textbf{#1}}}
\newcommand{\ImportTok}[1]{#1}
\newcommand{\InformationTok}[1]{\textcolor[rgb]{0.56,0.35,0.01}{\textbf{\textit{#1}}}}
\newcommand{\KeywordTok}[1]{\textcolor[rgb]{0.13,0.29,0.53}{\textbf{#1}}}
\newcommand{\NormalTok}[1]{#1}
\newcommand{\OperatorTok}[1]{\textcolor[rgb]{0.81,0.36,0.00}{\textbf{#1}}}
\newcommand{\OtherTok}[1]{\textcolor[rgb]{0.56,0.35,0.01}{#1}}
\newcommand{\PreprocessorTok}[1]{\textcolor[rgb]{0.56,0.35,0.01}{\textit{#1}}}
\newcommand{\RegionMarkerTok}[1]{#1}
\newcommand{\SpecialCharTok}[1]{\textcolor[rgb]{0.81,0.36,0.00}{\textbf{#1}}}
\newcommand{\SpecialStringTok}[1]{\textcolor[rgb]{0.31,0.60,0.02}{#1}}
\newcommand{\StringTok}[1]{\textcolor[rgb]{0.31,0.60,0.02}{#1}}
\newcommand{\VariableTok}[1]{\textcolor[rgb]{0.00,0.00,0.00}{#1}}
\newcommand{\VerbatimStringTok}[1]{\textcolor[rgb]{0.31,0.60,0.02}{#1}}
\newcommand{\WarningTok}[1]{\textcolor[rgb]{0.56,0.35,0.01}{\textbf{\textit{#1}}}}
\usepackage{graphicx}
\makeatletter
\def\maxwidth{\ifdim\Gin@nat@width>\linewidth\linewidth\else\Gin@nat@width\fi}
\def\maxheight{\ifdim\Gin@nat@height>\textheight\textheight\else\Gin@nat@height\fi}
\makeatother
% Scale images if necessary, so that they will not overflow the page
% margins by default, and it is still possible to overwrite the defaults
% using explicit options in \includegraphics[width, height, ...]{}
\setkeys{Gin}{width=\maxwidth,height=\maxheight,keepaspectratio}
% Set default figure placement to htbp
\makeatletter
\def\fps@figure{htbp}
\makeatother
\setlength{\emergencystretch}{3em} % prevent overfull lines
\providecommand{\tightlist}{%
  \setlength{\itemsep}{0pt}\setlength{\parskip}{0pt}}
\setcounter{secnumdepth}{-\maxdimen} % remove section numbering
\usepackage{float}
\usepackage{amsmath}
\usepackage{dplyr}
\floatplacement{figure}{H}
\ifLuaTeX
  \usepackage{selnolig}  % disable illegal ligatures
\fi
\IfFileExists{bookmark.sty}{\usepackage{bookmark}}{\usepackage{hyperref}}
\IfFileExists{xurl.sty}{\usepackage{xurl}}{} % add URL line breaks if available
\urlstyle{same}
\hypersetup{
  pdftitle={Statistical Learning in Movies},
  pdfauthor={Max Sellers, Claire Martino, Ari Augustine, and Rim Nassiri},
  hidelinks,
  pdfcreator={LaTeX via pandoc}}

\title{Statistical Learning in Movies}
\author{Max Sellers, Claire Martino, Ari Augustine, and Rim Nassiri}
\date{}

\begin{document}
\maketitle

\[H_T: \textrm{There is not difference in Average Wi-Fi speed between Hanson Hall of Science and Old Main.}\]
\[H_A: \textrm{There is a difference between in average Wi-Fi speeds between Hanson Hall of Science and Old main.}\]
\[\alpha=0.05\]

\hypertarget{abstract}{%
\subsubsection{Abstract}\label{abstract}}

\hypertarget{i.-introduction}{%
\subsection{I. Introduction}\label{i.-introduction}}

\hypertarget{ii.-methods}{%
\subsection{II. Methods}\label{ii.-methods}}

\hypertarget{a.-cleaning-data}{%
\subsubsection{a. Cleaning Data}\label{a.-cleaning-data}}

\begin{Shaded}
\begin{Highlighting}[]
\CommentTok{\#movies\_data \%\textgreater{}\%}
  \CommentTok{\#group\_by(title) \%\textgreater{}\%}
  \CommentTok{\#filter(budget != 0 \& revenue != 0 \& vote\_count \textgreater{} 100) \%\textgreater{}\%}
  \CommentTok{\#arrange(desc(vote\_average))}
\end{Highlighting}
\end{Shaded}

\hypertarget{b.-choosing-important-values-on-dataset}{%
\subsubsection{b. Choosing Important Values on
Dataset}\label{b.-choosing-important-values-on-dataset}}

\begin{Shaded}
\begin{Highlighting}[]
\NormalTok{budget }\OtherTok{=}\NormalTok{ movies\_data}\SpecialCharTok{$}\NormalTok{budget}
\NormalTok{popularity }\OtherTok{=}\NormalTok{ movies\_data}\SpecialCharTok{$}\NormalTok{popularity}
\NormalTok{revenue }\OtherTok{=}\NormalTok{ movies\_data}\SpecialCharTok{$}\NormalTok{revenue}
\NormalTok{genre }\OtherTok{=}\NormalTok{ movies\_data}\SpecialCharTok{$}\NormalTok{genre}

\CommentTok{\#number of cateogories and varaibles in the dataset}
\NormalTok{catagories }\OtherTok{=} \FunctionTok{ncol}\NormalTok{(movies\_data)}
\NormalTok{var }\OtherTok{=} \FunctionTok{nrow}\NormalTok{(movies\_data)}

\CommentTok{\#means of variables that will be using in analysis}
\NormalTok{mean\_budget }\OtherTok{=} \FunctionTok{round}\NormalTok{(}\FunctionTok{mean}\NormalTok{(budget),}\DecValTok{2}\NormalTok{)}
\NormalTok{mean\_popularity }\OtherTok{=} \FunctionTok{round}\NormalTok{(}\FunctionTok{mean}\NormalTok{(popularity),}\DecValTok{2}\NormalTok{)}
\NormalTok{mean\_revenue }\OtherTok{=} \FunctionTok{round}\NormalTok{(}\FunctionTok{mean}\NormalTok{(revenue),}\DecValTok{2}\NormalTok{)}
\end{Highlighting}
\end{Shaded}

\hypertarget{c.-training-data}{%
\subsubsection{c.~Training Data}\label{c.-training-data}}

\hypertarget{d.-testing-data}{%
\subsubsection{d.~Testing Data}\label{d.-testing-data}}

\hypertarget{iii.-results}{%
\subsection{III. Results}\label{iii.-results}}

\hypertarget{a.-graphs-of-specific-data}{%
\subsubsection{a. Graphs of Specific
Data}\label{a.-graphs-of-specific-data}}

\hypertarget{b.-linear-regression-summary-and-line}{%
\subsubsection{b. Linear Regression Summary and
Line}\label{b.-linear-regression-summary-and-line}}

\begin{Shaded}
\begin{Highlighting}[]
\CommentTok{\#linear regression model of budget vs vote\_average}
\NormalTok{linear\_1 }\OtherTok{=} \FunctionTok{lm}\NormalTok{(budget }\SpecialCharTok{\textasciitilde{}}\NormalTok{ vote\_average, }\AttributeTok{data =}\NormalTok{ movies\_data)}
\FunctionTok{summary}\NormalTok{(linear\_1)}
\end{Highlighting}
\end{Shaded}

\begin{verbatim}
## 
## Call:
## lm(formula = budget ~ vote_average, data = movies_data)
## 
## Residuals:
##       Min        1Q    Median        3Q       Max 
##  -7209200  -4921806  -4249044  -2893557 375212746 
## 
## Coefficients:
##              Estimate Std. Error t value Pr(>|t|)    
## (Intercept)    481572     255308   1.886   0.0593 .  
## vote_average   672763      42972  15.656   <2e-16 ***
## ---
## Signif. codes:  0 '***' 0.001 '**' 0.01 '*' 0.05 '.' 0.1 ' ' 1
## 
## Residual standard error: 17460000 on 44983 degrees of freedom
## Multiple R-squared:  0.005419,   Adjusted R-squared:  0.005397 
## F-statistic: 245.1 on 1 and 44983 DF,  p-value: < 2.2e-16
\end{verbatim}

\begin{Shaded}
\begin{Highlighting}[]
\CommentTok{\#graph of above data}
\FunctionTok{plot}\NormalTok{(budget}\SpecialCharTok{\textasciitilde{}}\NormalTok{ vote\_average, }\AttributeTok{data=}\NormalTok{ movies\_data, }\AttributeTok{xlab =} \StringTok{"Voter Average"}\NormalTok{, }\AttributeTok{ylab =} \StringTok{"Budget"}\NormalTok{, }\AttributeTok{main =} \StringTok{"Scatterplot of Budget vs Voter Average"}\NormalTok{)}
\FunctionTok{abline}\NormalTok{(linear\_1, }\AttributeTok{col =} \StringTok{"red"}\NormalTok{)}
\end{Highlighting}
\end{Shaded}

\includegraphics{IMDB_files/figure-latex/unnamed-chunk-5-1.pdf}

\begin{Shaded}
\begin{Highlighting}[]
\FunctionTok{plot}\NormalTok{(linear\_1)}
\end{Highlighting}
\end{Shaded}

\includegraphics{IMDB_files/figure-latex/unnamed-chunk-5-2.pdf}
\includegraphics{IMDB_files/figure-latex/unnamed-chunk-5-3.pdf}
\includegraphics{IMDB_files/figure-latex/unnamed-chunk-5-4.pdf}
\includegraphics{IMDB_files/figure-latex/unnamed-chunk-5-5.pdf}

\begin{Shaded}
\begin{Highlighting}[]
\CommentTok{\#linear regression model of revenue vs popularity}
\NormalTok{linear\_2 }\OtherTok{=} \FunctionTok{lm}\NormalTok{(revenue }\SpecialCharTok{\textasciitilde{}}\NormalTok{ popularity, }\AttributeTok{data =}\NormalTok{ movies\_data)}
\FunctionTok{summary}\NormalTok{(linear\_2)}
\end{Highlighting}
\end{Shaded}

\begin{verbatim}
## 
## Call:
## lm(formula = revenue ~ popularity, data = movies_data)
## 
## Residuals:
##        Min         1Q     Median         3Q        Max 
## -1.816e+09 -8.980e+06 -2.724e+05  3.309e+06  1.901e+09 
## 
## Coefficients:
##             Estimate Std. Error t value Pr(>|t|)    
## (Intercept) -4660716     292419  -15.94   <2e-16 ***
## popularity   5439042      43626  124.67   <2e-16 ***
## ---
## Signif. codes:  0 '***' 0.001 '**' 0.01 '*' 0.05 '.' 0.1 ' ' 1
## 
## Residual standard error: 55740000 on 44983 degrees of freedom
## Multiple R-squared:  0.2568, Adjusted R-squared:  0.2568 
## F-statistic: 1.554e+04 on 1 and 44983 DF,  p-value: < 2.2e-16
\end{verbatim}

\begin{Shaded}
\begin{Highlighting}[]
\FunctionTok{plot}\NormalTok{(linear\_2)}
\end{Highlighting}
\end{Shaded}

\includegraphics{IMDB_files/figure-latex/unnamed-chunk-5-6.pdf}
\includegraphics{IMDB_files/figure-latex/unnamed-chunk-5-7.pdf}
\includegraphics{IMDB_files/figure-latex/unnamed-chunk-5-8.pdf}
\includegraphics{IMDB_files/figure-latex/unnamed-chunk-5-9.pdf}

\begin{Shaded}
\begin{Highlighting}[]
\FunctionTok{plot}\NormalTok{(revenue }\SpecialCharTok{\textasciitilde{}}\NormalTok{ popularity, }\AttributeTok{data=}\NormalTok{ movies\_data, }\AttributeTok{xlab =} \StringTok{"Popularity"}\NormalTok{, }\AttributeTok{ylab =} \StringTok{"Revenue"}\NormalTok{, }\AttributeTok{main =} \StringTok{"Scatterplot of Revenue vs Popularity"}\NormalTok{)}
\FunctionTok{abline}\NormalTok{(linear\_2, }\AttributeTok{col =} \StringTok{"purple"}\NormalTok{)}
\end{Highlighting}
\end{Shaded}

\includegraphics{IMDB_files/figure-latex/unnamed-chunk-5-10.pdf}

\hypertarget{c.-multiple-linear-regression-summary}{%
\subsubsection{c.~Multiple Linear Regression
Summary}\label{c.-multiple-linear-regression-summary}}

\begin{Shaded}
\begin{Highlighting}[]
\NormalTok{multi\_linear }\OtherTok{=} \FunctionTok{lm}\NormalTok{(revenue }\SpecialCharTok{\textasciitilde{}}\NormalTok{ budget }\SpecialCharTok{+}\NormalTok{ vote\_average, }\AttributeTok{data =}\NormalTok{ movies\_data)}
\FunctionTok{summary}\NormalTok{(multi\_linear)}
\end{Highlighting}
\end{Shaded}

\begin{verbatim}
## 
## Call:
## lm(formula = revenue ~ budget + vote_average, data = movies_data)
## 
## Residuals:
##        Min         1Q     Median         3Q        Max 
## -632224373    -627029     402057    1524697 2116200015 
## 
## Coefficients:
##                Estimate Std. Error t value Pr(>|t|)    
## (Intercept)  -6.015e+06  6.041e+05  -9.958   <2e-16 ***
## budget        2.831e+00  1.116e-02 253.819   <2e-16 ***
## vote_average  9.355e+05  1.019e+05   9.177   <2e-16 ***
## ---
## Signif. codes:  0 '***' 0.001 '**' 0.01 '*' 0.05 '.' 0.1 ' ' 1
## 
## Residual standard error: 41310000 on 44982 degrees of freedom
## Multiple R-squared:  0.5918, Adjusted R-squared:  0.5917 
## F-statistic: 3.26e+04 on 2 and 44982 DF,  p-value: < 2.2e-16
\end{verbatim}

\begin{Shaded}
\begin{Highlighting}[]
\CommentTok{\#confidence interval insert here }
\end{Highlighting}
\end{Shaded}

\begin{Shaded}
\begin{Highlighting}[]
\CommentTok{\#logit.model.test = glm(revenue \textasciitilde{} budget + vote\_average, }
                       \CommentTok{\#data = movies\_data,family = binomial(link = "logit"))}
\end{Highlighting}
\end{Shaded}

\hypertarget{iv.-discussion}{%
\subsection{IV. Discussion}\label{iv.-discussion}}

\end{document}
